\documentclass[12pt]{article}
\usepackage[utf8]{inputenc}
\usepackage{amsmath, amsthm, latexsym, amssymb, graphicx, bold-extra, mathrsfs, frcursive}

\usepackage[pdftex]{color}
\usepackage[T1]{fontenc}
\usepackage[light, largesmallcaps]{kpfonts}
\usepackage{bm}

% VN
\usepackage[vietnamese]{babel}

%Insert image
\usepackage{graphicx}

% Simplifies margin settings
\usepackage{geometry}
\geometry{margin=20mm}

% Table
\usepackage{array}
\newcolumntype{P}[1]{>{\centering\arraybackslash}p{#1}}

\title{Show password from cmd - windows}
\date{Ngày 2 tháng 3 năm 2020}
\begin{document}
	\maketitle
	\ \\
	Xem password cmd:
	\begin{enumerate}
		\item Tới ổ C (cd) / Mở console (C:\textbackslash >)
		\item Lệnh netsh (C:\textbackslash >netsh). Đợi 1 lát con trỏ trở về netsh>
		\item wlan show profile
		\item Trong các profile show ra, chọn cái cần (vd ở đây là SFR-2b68)
		\item Gõ lệnh : netsh>wlan show profile SFR-2b68 key = clear
		\item Password ở phần : « Contenu de la clé            : DF9XCUIXFWHQ »
		
	\end{enumerate}
	\ \\
	\begin{center}
			\begin{tabular}{|P{7 cm}|P{7 cm}|}
			\hline 
			\textbf{Commandes} & \textbf{Qu'est-ce que ça fait} \\ 
			\hline 
			netsh wlan show profils & Affiche la liste des profils Wi-Fi configurés sur le système. \\ 
			\hline 
			netsh wlan delete profile name = \textit{"nom du profil"} & Supprime le profil Wi-Fi spécifié \\ 
			\hline 
			netsh wlan delete profile name = * & Supprime tous les profils Wi-Fi configurés sur le système. \\ 
			\hline 
		\end{tabular}
	\end{center}

	\ \\
	\textbf{Note:} \textit{Nếu profile có chứa dấu cách (space, \% 20), hãy đặt vào dấu nháy kép ("Hung chu tich 709")}

	
	
\end{document}