\documentclass[12pt]{article}
\usepackage[utf8]{inputenc}
\usepackage{amsmath, amsthm, latexsym, amssymb, graphicx, bold-extra, mathrsfs, frcursive}

\usepackage[pdftex]{color}
\usepackage[T1]{fontenc}
\usepackage[light, largesmallcaps]{kpfonts}
\usepackage{bm}

% VN
\usepackage[vietnamese]{babel}

%Insert image
\usepackage{graphicx}

% Hyperlink
\usepackage{hyperref}

% Simplifies margin settings
\usepackage{geometry}
\geometry{margin=20mm}

% Table
\usepackage{array}
\newcolumntype{P}[1]{>{\centering\arraybackslash}p{#1}}

\title{Đưa phương trình về giữa trong bài báo 2 cột \\
	Restore Centering for Equations in Custom Class}
\date{Ngày 4 tháng 3 năm 2020}

\begin{document}
	
	\maketitle
	
	\section{Problème}
	Phương trình không thể căn giữa bằng "centering" hoặc "begin\{center\}".
	
	Lý do: do dùng custum class (tự viết)
	
	\section{Solution}
	Bỏ dòng -RequirePackage[tbtags,fleqn]\{amsmath\} đi. Vì package fleq làm ctrinh ko centering eqn được.
	
	\section{Tham khảo}
	
	by jhargis » Wed Feb 26, 2014 1:18 am
	
	Actually, I also just found the solution (well, sort of). I went into the svjour3 cls file and found where the equations in two column mode are default set "fleqn" (line 42 of the code I believe). Just delete the "fleqn" inside the curly brackets , {}, and you'll get centered equations. Again, this is in the svjour3 style file itself that this must be done.
	
	Good luck...
	
	\url{https://latex.org/forum/viewtopic.php?t=22656}
	
\end{document}