\documentclass[12pt]{article}
\usepackage[utf8]{inputenc}
\usepackage{amsmath, amsthm, latexsym, amssymb, graphicx, bold-extra, mathrsfs, frcursive}

\usepackage[pdftex]{color}
\usepackage[T1]{fontenc}
\usepackage[light, largesmallcaps]{kpfonts}
\usepackage{bm}

% VN
\usepackage[french]{babel}

%Insert image
\usepackage{graphicx}

% Simplifies margin settings
\usepackage{geometry}
\geometry{margin=20mm}

% Table
\usepackage{array}
\newcolumntype{P}[1]{>{\centering\arraybackslash}p{#1}}

\title{Code\_Aster Ubuntu}
\date{Ngày 4 tháng 3 năm 2020}

\begin{document}
	\maketitle

	gmsh : maillage visualisation
	sudo apt-get install gmsh
	
	cd /home/t-s/salome\_meca/V2019.0.3\_universal/
	
	. salome\_prerequisites\_root.sh 
	. salome\_prerequisites.sh 
	as\_run 
	
	vim .bashrc
	(modification par \textbf{allias} source\_aster = ‘. / path1 \& . / path2’ : ajout de l’environnement des variables)
	
	bash
	
	Pour exécuter code aster :
	
	source\_aster : fait connaître la variable code\_aster à l’ordinateur
	puis :
	as\_run path (file.export) \\
	
	\noindent\rule{18.5cm}{0.4pt}
	\linebreak
	
	Dans réalité : les ponts routiers se composent de 3 à 4 travées, cela est équivalent à une poutre à 3-4 travées (4 maximum)
	
	Voir fréquence propre : dans fichier « .mess », cherche « calcul modal »
	\linebreak
	
	Dans le modèle analytique, on calcule $\lambda$.
	
	On a:
	$$\lambda^4 = \frac{\rho A}{EI} \omega^2$$
	$$\Rightarrow \omega = \sqrt{\frac{EI}{\rho A}} \lambda ^2 = 2 \pi f $$
	$$\Rightarrow f = \frac{1}{2 \pi}.\sqrt{\frac{EI}{\rho A}} \lambda ^2 $$
	
	\noindent\rule{18.5cm}{0.4pt}
	\linebreak
	
	Dans le script code\_Aster de Véronique, elle met des inerties quadratiques $I_y$ et $I_z$ d'une poutre en section $I$. Dans le repère global, l'axe $s$ se situe le long de la poutre, l'axe $z$ est vertical, tend vers le haut. L'axe $y$ est dans la direction perpendiculaire de la poutre.
	
	Étant donné les deux inerties $I_y$ et $I_z$, $I_z < I_y$ (environ 10 fois plus faible), le logiciel cherche les fréquence propre dans les 2 sens (dans le plan$(0xz)$ et hors plan).
	
	Les résultat obtenue par la projection sur la base de déformés modales. Par orthogonalité, les modes hors plan seront annulé (on ne garde que les modes de vibration dans le plan $(Oxz)$).
	
	
\end{document}
